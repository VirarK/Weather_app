\documentclass[a4paper, 12pt, french]{article}
\usepackage{ae,lmodern}
\usepackage[french]{babel}
\usepackage[utf8]{inputenc}
\usepackage[T1]{fontenc}
\usepackage[dvipsnames]{xcolor}
\usepackage{graphicx}
\usepackage{hyphenat}
\usepackage[left=15mm, right=15mm]{geometry}
\usepackage{csquotes}
\usepackage{bookmark}
\usepackage{biblatex}
\usepackage{listings}
\usepackage{hyperref}
\usepackage{eurosym}

\addbibresource{rapport.bib}

\lstset{
	aboveskip=3mm,
	belowskip=-2mm,
	backgroundcolor=\color{lightgray},
	basicstyle=\footnotesize,
	breakatwhitespace=false,
	breaklines=true,
	captionpos=b,
	commentstyle=\color{ForestGreen},
	deletekeywords={\ldots},
	escapeinside={\%*}{*)},
	extendedchars=true,
	framexleftmargin=16pt,
	framextopmargin=3pt,
	framexbottommargin=6pt,
	frame=tb,
	keepspaces=true,
	keywordstyle=\color{blue},
	language=Python,
	literate=
	{²}{{\textsuperscript{2}}}1
	{⁴}{{\textsuperscript{4}}}1
	{⁶}{{\textsuperscript{6}}}1
	{⁸}{{\textsuperscript{8}}}1
	{€}{{\euro{}}}1
	{é}{{\'{e}}}1
	{è}{{\`{e}}}1
	{ê}{{\^{e}}}1
	{ë}{{\¨{e}}}1
	{É}{{\'{E}}}1
	{Ê}{{\^{E}}}1
	{û}{{\^{u}}}1
	{ù}{{\`{u}}}1
	{â}{{\^{a}}}1
	{à}{{\`{a}}}1
	{á}{{\'{a}}}1
	{ã}{{\~{a}}}1
	{Á}{{\'{A}}}1
	{Â}{{\^{A}}}1
	{Ã}{{\~{A}}}1
	{ç}{{\c{c}}}1
	{Ç}{{\c{C}}}1
	{õ}{{\~{o}}}1
	{ó}{{\'{o}}}1
	{ô}{{\^{o}}}1
	{Õ}{{\~{O}}}1
	{Ó}{{\'{O}}}1
	{Ô}{{\^{O}}}1
	{î}{{\^{i}}}1
	{Î}{{\^{I}}}1
	{í}{{\'{i}}}1
	{Í}{{\~{Í}}}1,
	morekeywords={*,self, \_\_init\_\_, \_\_eq\_\_, \_\_str\_\_},
	numbers=left,
	numbersep=10pt,
	numberstyle=\tiny\color{black},
	rulecolor=\color{black},
	showspaces=false,
	showstringspaces=false,
	showtabs=false,
	stepnumber=1,
	stringstyle=\color{ForestGreen},
	tabsize=4,
	title=\lstname,
}

\title{
	\Huge
	\textbf{Application météo}
	\vspace{0.4cm}

	\LARGE
	Application Web et Sécurité
}

\author{
	Melissa Allaoua \\
	LE DENMAT Mickaël \\
	Hasnae Gaizi \\
	Gabriel Scrève
}

\begin{document}
	\begin{titlepage}
    \begin{center}
        \vspace*{1cm}

        \Huge
        \textbf{Application météo}

        \vspace{0.4cm}
        \LARGE
        Application Web et Sécurité

        \vspace{1.6cm}

        \large
        Melissa Allaoua \\
        Mickaël Le Denmat \\
        Hasnae Gaizi \\
        Gabriel Scrève \\

        \vfill

        \vspace{0.8cm}
        \includegraphics[width=0.32\textwidth]{images/UVSQ-logo}

        \vspace{0.4cm}

        \Large
        Université de Versailles Saint-Quentin-en-Yvelines \\
        \vspace{0.4cm}
        DATE ???
    \end{center}
\end{titlepage}

	\newpage
	\renewcommand{\contentsname}{Table des matières}
	\tableofcontents

	\newpage
	\section{Application météo}
		Notre projet est une application météo. Elle affiche les informations classiques 
    	des applications météo comme une petite discription de la météo, la température,
		le ressenti, la vitesse du vent. Nous avons ajouté la températeur de 
   		la ville toutes les trois heures, et une prévisualisation de la météo sur sept jours.
		% TODO : refaire screen
		\scalebox{.33}{\includegraphics{images/app-meteo.PNG}}

		Pour rechercher la météo d'une ville, l'utilisateur rentre le nom de la ville et le pays
    	(en englais) et appuie sur le bouton avec le logo de loupe.
		L'utilisateur a la possibilité de se connecter et de s'inscrire à l'aide de son nom et de son
    	mail pour pouvoir ajouter la ville recherché en favoris et accéder à tous ces favoris via
    	la page de profile.

	\section{Technologie(s) utilisée(s)}
		\subsection{Bootstrap}
			Bootstrap est un framework css \cite*{Bootstrap} utilisé pour construire des sites
			réactif aux différentes tailles d'écrans ainsi que des sites pour les
			téléphones.

			Nous l'avons utilisé afin d'organiser rapidement et facilement
			les éléments au sein de notre application. Il nous a permis de
     		découper les différentes pages en grilles et placer nos éléments où non le
			souhaitons, utiliser les mise en forme de Bootstrop pour les boutons,
			les espaces entre les éléments, les couleurs, \ldots

			Le site de Bootstrap propose aussi beaucoup d'exemples \cite*{Bootstrap-Exemples}
			(de template) pour des éléments classiques dans les sites, comme
			les menues, les barres de navigation, les pages de login. 
			Nous avons en copié certains pour avoir une base fonctionnelle 
			que nous avons modifié pour
			ajouter ce dont nous avons besoin et que cela sois dans le même
			style que notre application.

			Beaucoup d'autres framework existent mais Bootstrap, étant la plus connue, a plus de
			documentations, de tutoreils Youtube ou autres et plus d'aide sur les forums en cas de
			bug. Du plus lors du developpement de l'application, Bootstrap a parfaitement
			collé à nos besoins et nous n'avons pas eu besoin de trouvé une autre bibliothéque.

		\subsection{OpenWeatherApp}
			OpenWeatherApp \cite*{OpenWeatherApp} est site proposant des services 
      		concernant la météo. Il permet
			de faire des requêtes à différentes API's pour connaître la météo actuelle,
			l'humidité, le ressenti, l'uv, Il indique aussi la météo heure par heure, 
			les prévisions de la semaine. Il existe aussi d'autre API spécifiques comme 
			pour la mer ou les pays avec les risques de neige. Enfin il propose aussi 
			une API de permmettant de connaître les informations d'une ville
			en fonction de ses coordonnées (latitude, longitude).

			Nous utilisons une de leur API appelé OneCall \cite*{OpenWeatherApp-OneCall}
			qui nous permet d'obtenir toutes les informations essentiels en un seul appel.
			Nous récupèrons un json avec la météo actuelle, une prévision de la météo pour
			les sept jours a venir, une prévision pour les heurs a venir. L'appel a l'API se
			fait comme cela :\url{https://api.openweathermap.org/data/2.5/onecall?lat=
			{lat}&lon={lon}&exclude={part}&appid={API key}}.

			Ensuite nous utilisons une variante de cette dernière afin d'avoir les informations
			concernant la météo des heures passées. Pour cela nous faisons appel à 
			\url{https://api.openweathermap.org/data/2.5/onecall/
			timemachine?lat={lat}&lon={lon}&dt={time}&appid={API key}}.

			Avec ces deux API's nous avons toutes les informations nécéssaires à notre projet.

			
		\subsection{Express et NodeJS}
			% TODO : Gabriel

		\subsection{MySQL}
			MySQL est un système de gestion de bases de données \cite*{MySQL-wikipedia},
			qui % TODO : à finir

	\section{Quelques point(s) important(s)}


	\section{Structure}


	\printbibliography

\end{document}
